\chapter{Depth, Codimension and Cohen-Macaulay Rings}
Same as the previous chapter, all rings are assumed to be Noetherian.
\section{Depth}

Let \(R\) be a ring, and \(I\subset R\) is an ideal. \(M\) is a finitely generated \(R\)-module such that \(IM\neq M\). Recall from the last chapter that the depth \(\dep(I,M)\) is the length of the maximal \(M\)-sequence. It can be charaterized in terms of the vanishing of the homology of the Koszul complex. We want to explore the behavior of depth under localization.

\begin{lemma}\label{deplocal}
    \(R\) is a ring and \(P\) is a prime ideal in the support of a finitely generated \(R\)-module \(M\), then any \(M\)-sequence in \(P\) localizes to a \(M_P\)-sequence. Thus for any ideal \(I\subset P\), we have \(\dep(I,M)\leq \dep (I_P,M_P)\), the latter taken in the ring \(R_P\). In general, the inequality may be strict, but for any ideal \(I\) there exists a maximal ideal \(P\) in the support of \(M\) such that \(\dep(I,M)=\dep(I_P,M_P)\). In particular, if \(P\) is a maximal ideal, then \(\dep(P,M)=\dep(P_P,M_P)\).
\end{lemma}

\begin{proof}
    
\end{proof}

There is a lemma for the depth similar to the princiapl ideal theorem of codimension.

\begin{lemma}\label{depprinc}
   Let \((R,\mathfrak{m})\) be a local ring. \(M\) is a finitely generated \(R\)-module, \(I\subset R\) is an ideal and \(y\in \mathfrak{m}\). Then 
   \[\dep(I+(y),M)\leq \dep (I,M)+1.\] 
\end{lemma}

\begin{proof}
    
\end{proof}

\section{Depth and the Vanishing of Ext}

\begin{proposition} 
     Let \(R\) be a ring and \(M,N\) be finitely generated \(R\)-modules. If \(\ann M+\ann N=R\), then \(\ext_R^r(M,N)=0\) for every \(r\). Otherwise, \(\dep(\ann M,N)\) is the smallest number \(r\) such that \(\ext_R^r(M,N)\neq 0\).
\end{proposition}

\begin{proof}
    Note that if \(s\in \ann M\) or \(s\in \ann N\), then \(s\) also annihilates \(\hom_R(M,N)\). This tells us that \(s\) annihilates \(\ext^r_R(M,N)\) for any \(r\).

    \begin{claim}
      \(\ann M+\ann N=R\) if and only if \(\ann(M)N=N\).
    \end{claim}
    \begin{claimproof}
      Suppose \(\ann(M)N=N\). By Nakayama's lemma, there exists \(r\in R\) such that \((1-r)N=0\). Thus \(1-r\in \ann N\), and \(1\in \ann M+\ann N\). This proves \(\ann M+\ann N=R\). Conversely, suppose \(\ann M+\ann N=R\). We can write \(1=r+s\) where \(r\in \ann M\) and \(s\in \ann N\), hence 
      \[rN=(r+s)N=N.\]
      This implies that \(\ann(M)N=N\).
    \end{claimproof}

    Suppose \(\ann M+\ann N\neq R\). This means that \(\ann (M)N\neq N\), so the depth 
    \[d=\dep (\ann M,N)<\infty.\]
    We do induction on \(d\). If \(d=0\), We need to show that 
    \[\ext^0_R(M,N)=\hom_R(M,N)\neq 0.\]
    Since \(\dep(\ann M,N)=0\), this means that \(\ann M\) are all zero divisors for \(N\), so \(\ann M\) is contained in one of the associated primes \(\mathfrak{p}\) of \(N\). Note that for finitely generated modules \(M,N\), localization commutes with taking hom, so it is enough to prove this after localizing at \(\mathfrak{p}\). Note that \(\mathfrak{p}=\ann_R(m)\) for some \(m\in N\), so the localized \(N_\mathfrak{p}\) must contain a copy of the field \(R_\mathfrak{p}/\mathfrak{p}R_\mathfrak{p}\) generated by \(m\) in \(R_\mathfrak{p}\). On the other hand, we know that \(M_\mathfrak{p}\neq 0\), by Nakayama's lemma, \(M_\mathfrak{p}/\mathfrak{p}M_\mathfrak{p}\neq 0\) as a vector space over \(R_\mathfrak{p}/\mathfrak{p}R_\mathfrak{p}\). This implies we have a nonzero homomorphism from \(M_\mathfrak{p}\) to \(N_\mathfrak{p}\).

    Now assume \(d\geq 1\) and let \(x\in \ann M\) be a nonzero divisor on \(N\). We have \(\ann(M)(N/xN)\neq N/xN\) and \(\dep(\ann M,N/xN)=d-1\). By induction \(\ext^{d-1}_R(M,N/xN)\neq 0\), and 
    \[\ext^{<d-1}_R(M,N/xN)=0.\]
    Consider the short exact sequence
    \[0\rightarrow N\xrightarrow{x}N\rightarrow N/xN\rightarrow 0.\]
    Apply \(\ext_R(M,-)\) and we obtain a long exact sequence. Note that \(x\in \ann M\), so it annihilates \(\ext^j_R(M,N)\). The induced map will be the zero map. Thus, we get a collection of short exact sequence
    \[0\rightarrow \ext^{j-1}_R(M,N)\rightarrow \ext^{j-1}_R(M,N/xN)\rightarrow \ext^j_R(M,N)\rightarrow 0\]
    for all \(j\geq 1\). By induction on \(d\), we can see that 
    \[\ext^i_R(M,N)=0\]
    for all \(i<d\) and 
    \[\ext^d_R(M,N)\cong \ext^{d-1}_R(M,N/xN)\neq 0.\]
\end{proof}

\begin{example}
    Let \(k\) be a field and \(R=k[x,y,z]\). Consider the ideal \(I=(xy,yz)=(x,z)\cap (y)\). We have a free resolution for \(R/I\):
    \[0\leftarrow R/I\leftarrow R\xleftarrow{(xy\ \ \ yz)} R^2\xleftarrow{\begin{pmatrix} -z\\ x\end{pmatrix}}R\leftarrow 0.\]
    The \(\ext_R(R/I,R)\) can be computed from the following complex 
    \[R\xrightarrow{\begin{pmatrix} xy\\ yz \end{pmatrix}} R^2\xrightarrow{(-z\ \ \ x)}R\]
    So \(\hom(R/I,R)=0\), \(\ext^1_R(R/I,R)=R/(y)\) and \(\ext^2_R(R/I,R)=R/(x,z)\).
\end{example}

\begin{example}
    Let \(x_1,\ldots,x_n\) be a regular sequence in \(R\). The Koszul complex \(K(x_1,\ldots,x_n)\) is a free resolution of \(R/(x_1,\ldots,x_n)\). Write \(N=R/(x_1,\ldots,x_n)\). We know that \(\ext_R(N,M)\) can be computed via the homology of the complex \(\hom_R(K(x_1,\ldots,x_n),M)\). Recall that the Koszul complex is isomorphic to its dual, we have 
    \[\hom_R(K(x_1,\ldots,x_n),M)\cong M\otimes_R K(x_1,\ldots,x_n).\]
    And we recover a previous theorem. On the other hand, \(\dep I=1\) as \(xy\in I\) is a nonzero divisor in \(R\), for any \(x\in I\), \(x\) is a zero divisor in \(R/(xy)\). 
\end{example}

\begin{corollary}
  For any nonzero module \(M\), \(\mathrm{pd}_R M\geq \dep \ann M\) where \(\mathrm{pd}_R M\) is the projective dimension of \(M\).
\end{corollary}

\begin{proof}
    Take \(N=R\).
\end{proof}

\section{Cohen-Macaulay Rings}
Suppose \((R,\mathfrak{m})\) is a regular local ring (that is, the number of generators of \(\mathfrak{m}\) equals \(\dim R\)), then any minimal set of generators for \(\mathfrak{m}\) is a regular sequence. And we have \(\dep \mathfrak{m}=\codim \mathfrak{m}\). This equality also holds in some other rings.

\begin{theorem}
  Let \(R\) be a ring such that \(\dep \mathfrak{m}=\codim \mathfrak{m}\) for every maximal ideal \(\mathfrak{m}\). If \(I\subset R\) is a proper ideal, then \(\dep I=\codim I\).
\end{theorem}

\begin{proof}
    We have already proved that \(\dep I\leq \codim I\). We need to prove \(\dep I\geq \codim I\). 

    We know \(I\subset \mathfrak{m}\), so localizing at \(\mathfrak{m}\) does not change \(\codim I\). Similarly, by lemma \ref{deplocal}, localizing also does not change \(\dep I\). We may assume \((R,\mathfrak{m})\) is a local ring with \(I\subset \mathfrak{m}\). If \(\sqrt{I}=\mathfrak{m}\), then \(\codim I=\codim \mathfrak{m}\), and by Corollary \ref{depgeo}, we have 
    \[\dep I=\dep \sqrt{I}=\dep \mathfrak{m}=\codim \mathfrak{m}=\codim I.\]
    Now suppose \(\sqrt{I}\subsetneq \mathfrak{m}\). By Noetherian induction, we may assume the theorem holds for all ideals strictly larger than \(I\).  Since \(\mathfrak{m}\) is not a minimal prime over \(I\), by prime avoidance, there exists \(x\in \mathfrak{m}\) such that \(x\) is not in any minimal prime over \(I\). Thus, by induction and Lemma \ref{depprinc}, we have 
    \[\dep I+1\geq \dep(I+(x))=\codim (I+(x))=\codim I+1.\]
    This implies that 
    \[\dep I\geq \codim I.\]
\end{proof}

Rings satisfying the hypothesis in the above theorem have a special name: Cohen-Macaulay rings.

\begin{definition}[Cohen-Macaulay rings]
     A ring such that \(\dep \mathfrak{m}=\codim \mathfrak{m}\) for every maximal ideal \(\mathfrak{m}\) is called a \textbf{Cohen-Macaulay} ring. 
\end{definition}

The property of Cohen-Macaulay is local in some sense and can be passed to polynomial rings. Below all Cohen-Macaulay will be abbreviated as CM.

\begin{proposition}\label{CMlocal}
     \(R\) is CM iff \(R_\mathfrak{m}\) is CM for every maximal ideal \(\mathfrak{m}\) of \(R\), and then \(R_\mathfrak{p}\) is CM for every prime ideal \(\mathfrak{p}\) of \(R\). A local ring is CM iff its completion if CM.
\end{proposition}

\begin{proof}
    Suppose \(R\) is CM and \(\mathfrak{p}\subset R\) is a prime idea. By lemma \ref{deplocal}, we have 
    \[\codim \mathfrak{p}_\mathfrak{p}=\codim \mathfrak{p}=\dep \mathfrak{p}\leq \dep \mathfrak{p}_\mathfrak{p}\leq \codim \mathfrak{p}_\mathfrak{p}.\]
    They are all equalities so \(R_\mathfrak{p}\) is CM. Conversely, if \(R_\mathfrak{m}\) is CM for every maximal ideal \(\mathfrak{m}\), then by lemma \ref{deplocal}, we have 
    \[\dep \mathfrak{m}=\dep \mathfrak{m}_\mathfrak{m}=\codim \mathfrak{m}_\mathfrak{m}=\codim \mathfrak{m}.\]
    This proves that \(R\) is also CM.

    Let \((R,\mathfrak{m})\) be a local ring and \((\hat{R},\hat{\mathfrak{m}})\) be its completion. We know that \(\codim \mathfrak{m}=\codim \hat{\mathfrak{m}}\) since \(\dim R=\dim \hat{R}\). Next, we want to show that \(\dep (\mathfrak{m},R)=\dep (\hat{\mathfrak{m}}, \hat{R})\). Let \(x_1,\ldots,x_n\) be generators of \(\mathfrak{m}\) and \(K(x_1,\ldots,x_n)\) be the Koszul complex of the ideal \(\mathfrak{m}\) in \(R\). We know that \(\hat{K}=\hat{R}\otimes_R K(x_1,\ldots,x_n)\) is the Koszul complex of \(\hat{\mathfrak{m}}\) in \(\hat{R}\), and 
    \[H^*(\hat{K})=\hat{R}\otimes_R K(x_1,\ldots,x_n).\]
    Note that \(R\rightarrow \hat{R}\) is faithfully flat for a local ring \(R\). This implies 
    \[\dep (\mathfrak{m},R)=\dep (\hat{\mathfrak{m}}, \hat{R}).\]
\end{proof}

\begin{proposition}[CM can pass to polynomial rings]\label{CMpoly}
     A ring \(R\) is CM iff the polynomial ring \(R[x]\) is CM.
\end{proposition}

\begin{proof}
    If \(R[x]\) is CM, and we know that \(x\) is a nonzerodivisor, so \(R=R[x]/(x)\) is also CM.
    
    Suppose \(R\) is CM. It is suffice to prove \(R[x]_\mathfrak{m}\) is CM for every maximal ideal \(\mathfrak{m}\subset R[x]\). Let \(\mathfrak{m}\) be a maximal ideal and \(\mathfrak{n}=\mathfrak{m}\cap R\). The complement of \(\mathfrak{n}\) in \(R\) is contained in the complement of \(\mathfrak{m}\) in \(R[x]\), we have 
    \[R[x]_\mathfrak{m}=(R_\mathfrak{n}[x])_\mathfrak{m}.\]
    So we may assume \(R\) is a local ring with the maximal ideal \(\mathfrak{n}\). Note that
    \[R[x]/\mathfrak{n}R[x]=(R/\mathfrak{n})[x]\]
    is a PID. So modulo \(\mathfrak{n}\) the ideal \(\mathfrak{m}\) is generated by a monic polynomial \(f(x)\). Let \(x_1,\ldots,x_n\) be an \(R\)-sequence in \(\mathfrak{n}\), then it is also an \(R[x]\)-sequence in \(\mathfrak{m}\) since \(R[x]\) is a free \(R\)-module. Moreover, note that \(f(x)\) is a nonzerodivisor in \((R/I)[x]\) for any ideal \(I\subset R\), so 
    \[x_1,\ldots,x_n,f(x)\]
    is an \(R[x]\)-sequence in \((\mathfrak{m})\), and this implies that 
    \[\dep(\mathfrak{m},R[x])\geq \dep (\mathfrak{n},R)+1.\]
    On the other hand, by the principal ideal theorem and \(R\) is CM, we have 
    \[\codim \mathfrak{m}\leq \codim \mathfrak{n}+1=\dep (\mathfrak{n}, R)+1\leq \dep (\mathfrak{m},R[x]).\]
    Thus, \(R[x]\) is CM.  
\end{proof}

In the next part, we discuss some nice properties and applications for CM rings. 

\begin{definition}[Catenary]
     A ring \(R\) is \textbf{catenary}, or has the saturated chain condition, if given any prime ideals \(\mathfrak{p}\subset \mathfrak{q}\) of \(R\), the maximal chains of prime ideals between \(\mathfrak{p}\) and \(\mathfrak{q}\) all have the same length. \(R\) is \textbf{universally catenary} if every finitely generated \(R\)-algebras is catenary.  
\end{definition}

\begin{corollary}\label{CMcatenary}
  CM rings are universally catenary. Moreover, in a local CM ring, any two maximal chains of prime ideals have equal length, and every associated prime of \(R\) is minimal. 
\end{corollary}

\begin{proof}
    By \ref{CMpoly}, we know that the polynomial ring over a CM ring is still CM. Let \(R\) be a CM ring. To prove any finitely generated \(R\)-algebra is CM, it is thus suffice to prove that any holomorphic image \(S\) of \(R\) is catenary. Any two maximal chains of prime ideals in \(S\) can be pulled back to two maximal chains of prime ideals \(\mathfrak{p}\subset \mathfrak{q}\) in \(R\), and by \ref{CMlocal}, we may localize \(R\) at the prime ideal \(\mathfrak{q}\) without changing the chains. So we may assume \(R\) is a local CM ring, and the first statement follows from the second. 

   second
\end{proof}

\begin{definition}[Equidimensional]
     A ring \(R\) is \textbf{equidimensional} if all its maximal ideals have the same codimension and all its minimal prime ideals have the same codimension.
\end{definition}

For a CM ring \(R\), the second condition in the definition follows from the first by Corollary \ref{CMcatenary}. If \(R_1\) and \(R_2\) are CM, the direct product \(R_1\times R_2\) is also CM. So a CM ring need not be equidimensional, but we have the following:

\begin{corollary}
  Any local CM ring is equidimensional.
\end{corollary}

Geometrically speaking, if a variety \(X\) is locally CM at a point \(p\) (in the sense that the local ring \(\mathcal{O}_{X,p}\) is CM), then \(p\) cannot lie on two components of different dimensions. For example, the ring \(\mathbb{C}[x,y,z]/(xy,yz)\) is not CM at the origin. We can get a lot of information about an ideal \(I\) if we know the ring \(R/I\) is CM.

\begin{proposition}
     Let \(R\) be a CM ring. If \(I=(x_1,\ldots,x_n)\) is an ideal generated by \(n\) elements in a CM ring \(R\) such that \(\codim I=n\), the largest possible value, then \(R/I\) is a CM ring. 
\end{proposition}

\begin{proof}
    
\end{proof}

\begin{corollary}[Unmixedness Theorem]
  Let \(R\) be a ring. If \(I=(x_1,\ldots,x_n)\) is an ideal generated by \(n\) elements such that \(\codim I=n\), then all minimal primes of \(I\) have codimension \(n\). If \(R\) is CM, then every associated prime of \(I\) is minimal over \(I\). 
\end{corollary}